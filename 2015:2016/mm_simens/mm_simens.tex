\documentclass[english]{article}

\usepackage{babel}
\usepackage{graphicx}
\usepackage{times}
\usepackage{pifont}
\usepackage[margin=1in]{geometry}
\usepackage{eurosym}
\usepackage{fancyhdr}
\usepackage[hidelinks]{hyperref}

\pagestyle{fancy}
\fancyhf{}


%HEADER
%**************************************************************************************
\pagestyle{fancy}
\fancyhf{}
%**************************************************************************************
\lhead{Microsensors and Mechanics}		 	 
\rhead{OPC client for sensor data} 
\lfoot{EFA12SF}
\rfoot{Alexey Tukalo}
%**************************************************************************************

\date{}
\setlength\parindent{0pt}

\begin{document}

Sir, I am very sorry for the delay, I found the deadline only today and only by accident, because the assignments not are placed in Moodle in chronological order of the deadlines. So I check the deadlines of tasks which are upper and they are far away, so I did not even expect that I skipped this one which is on the bottom of page.
\section{What is OPC?}

The OPC standard is a series of specifications developed by industry vendors, end-users and software developers. These specifications define the interface between Clients and Servers, as well as Servers and Servers, including access to real-time data, monitoring of alarms and events, access to historical data and other applications.[1]\\\\

The sensors can transmit the data to OPS via protocols like: Web Socket, Web Service and HTTP. Web Service have the best support from tools available for Java and .NET.[2]

\section{Toolkits}

There is various number of open-source and commercial toolkits for different platforms. The commune commercial APIs support C, C++, Java and .NET and the open-source solution are exist for C, C++, Java, JavaScript(node) and Python.[3]\\

For example:
\begin{enumerate}
\item  C++ OpenSource OPC UA Application Server and an OPC UA Web Server
\item Number of SDK which combines JAVA or C\# code with encapsulation of open-source framework for ANSI C 
\end{enumerate}

It is very important to be sure that the SDK is certified by OPC Foundation.


\section{Data-management}

It is possible to use any data management tools and databases available for this platform, so developer can keep the OPC data in any type of SQL or even NoSQL databases like:
For example:
\begin{enumerate}
\item MS SQL Server
\item MySQL
\item PostgreSQL
\item MongoDB (NoSQL)
\end{enumerate}

\section{References}
\begin{enumerate}
\item $opcfoundation.org/about/what-is-opc/ $ (03.11.15)
\item $en.wikipedia.org/wiki/Open_Platform_Communications  $ (03.11.15)
\item $en.wikipedia.org/wiki/OPC_Unified_Architecture$ (03.11.15)
\end{enumerate}



\end{document}
