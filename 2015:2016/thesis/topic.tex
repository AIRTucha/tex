\documentclass[english]{article}

\usepackage{babel}
\usepackage{graphicx}
\usepackage{times}
\usepackage{pifont}
\usepackage[margin=1in]{geometry}
\usepackage{eurosym}
\usepackage{fancyhdr}
\usepackage[hidelinks]{hyperref}

\pagestyle{fancy}
\fancyhf{}


%HEADER
%**************************************************************************************
\pagestyle{fancy}
\fancyhf{}
%**************************************************************************************
\lhead{Thesis}		 	 
\rhead{Topic Plan} 
\lfoot{EFA12SF}
\rfoot{Alexey Tukalo}
%**************************************************************************************

\date{}
\setlength\parindent{0pt}

\begin{document}

\title{\vspace{2in}Topic Plane\\
\small for Bachelor Thesis\\
\vspace{0.5in}\includegraphics{savonia.jpg}}

\nopagebreak
\maketitle


\vspace{3in}

\author{
\begin{flushright}
Alexey Tukalo,\\
EFA12SF,\\
Information Technology,\\
Savonia University of Applied Sciences
\end{flushright}
}

\date{\today}
\thispagestyle{empty}

\newpage
\setcounter{page}{1}
\setcounter{tocdepth}{2}



%MAIN CONTENT ******************************************************************************************************************

\section{Topic defenition}

The topic of my final thesis is "Implementation of volume rendering in LightningChart\footnote{C\# library for visualisation of scientific data developed in Arction}". I made myself familiar with a concept of volume rendering during my internship at Karlsruhe Institute of Technology(KIT), there I worked on modification of Tomoraycaset 2\footnote{WebGL based library for volume rendering developed by KIT} to give it an opportunity to render multimodal volume data. Right now, I am working at Arction on development of the LightningChart and my knowledge in volume rendering makes me suitable candidate for implementation of the feature in the library.\\

I look at data visualisation and computer graphics as at the main areas of my professional growth. I have already made a lot to become valuable specialist at the field, but it is only start of the journey. I hope that the work on volume rendering will help me to improve my knowledge, will teach me a lot about computer graphics and visualisation. Moreover, the project will perfectly fit to my portfolio.

\section{Application}

As I already noticed, the work will be done in collaboration with Arction Oy and the source code will become part of LightningChart library. The project will give new possibilities for current Arction's clients and hopefully attract new customers interested in volume rendering.\\

I will have to choose the best volume rendering technique and implement it inside the LightningChart for Direct 9 and Direct 11 version of the library.\\

Arction also develops custom visualizations for their clients' cases and one of the key partners needs the visualisation related with volume rendering, so my thesis work will be the core of the project in a future.

\section{Implementation}

I am going to start the work with a literature research. The main aim of the research is to find the fastest implementation of the volume rendering possible in modern graphic cards. After that I will start to implement the solution with C\# and HLSL inside the library and I will write the final report in the end of my work.\\ 

The project requires deep knowledge in object-oriented programming with C\#, in computer graphics and programing of shaders for graphic cards with HLSL. It do not require any special equipment except PC.

\section{Schedule}

I am going to start my literature research at the first of February and I am going to do it for two weeks, after that I will work on actual implementation of the source code for 2 months and I am going to write the paper during to last weeks of April. So, in according with my plan the report have to be ready at the first of May.

\end{document}
